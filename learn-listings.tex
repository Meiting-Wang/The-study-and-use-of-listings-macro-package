%%----------------------------导言区--------------------------%%
%2020年9月13日

\documentclass[UTF8,hyperref]{ctexart}
\usepackage[a4paper,scale=0.75]{geometry}
\usepackage[dvipsnames]{xcolor}
\usepackage{lipsum}
\usepackage{longtable} %默认下longtable本身会被看作一个整体水平居中的整段,且段前段后会有额外的间距,以表示强调
\usepackage{multirow}
\usepackage{shortvrb}
\usepackage{listings}
% Description: Stata language definition for listings
% Author: WANG Meiting, PhD, Institute for Economic and Social Research, Jinan University
% Email: wangmeiting92@gmail.com
% Created on Sep 15, 2020
%
% Inspired by "https://git.io/JU8tn" and "https://git.io/JU8tG"

\lstdefinelanguage{Stata}{
    % 首要关键字
    morekeywords=[1]{regress,reg,summarize,sum,display,di,generate,gen,bysort,use,import,delimited,predict,quietly,probit,margins,test,forvalues,foreach,set,sysuse,graph,local,global,export,twoway,replace,append,clear,if,else,dis,order,mat,matrix,mata,excel,cd,logit,aggregate,array,boolean,break,byte,case,catch,class,colvector,complex,const,continue,default,delegate,delete,do,double,eltypedef,end,enum,explicit,external,float,for,friend,function,goto,inline,int,long,namespace,new,numeric,NULL,operator,orgtypedef,pointer,polymorphic,pragma,private,protected,public,quad,real,return,rowvector,scalar,short,signed,static,strL,string,struct,super,switch,template,this,throw,transmorphic,try,typedef,typename,union,unsigned,using,vector,version,virtual,void,volatile,while,in,define,rown,rownames,coln,colnames,list,corr,ttest,tokenize},
    % 函数关键字
    morekeywords=[2]{bofd,Cdhms,Chms,Clock,clock,Cmdyhms,Cofc,cofC,Cofd,cofd,daily,date,day,dhms,dofb,dofC,dofc,dofh,dofm,dofq,dofw,dofy,dow,doy,halfyear,halfyearly,hh,hhC,hms,hofd,hours,mdy,mdyhms,minutes,mm,mmC,mofd,month,monthly,msofhours,msofminutes,msofseconds,qofd,quarter,quarterly,seconds,ss,ssC,tC,tc,td,th,tm,tq,tw,week,weekly,wofd,year,yearly,yh,ym,yofd,yq,yw,abs,ceil,cloglog,comb,digamma,exp,expm1,floor,int,invcloglog,invlogit,ln,ln1m,ln,ln1p,ln,lnfactorial,lngamma,log,log10,log1m,log1p,max,min,mod,reldif,round,sign,sqrt,trigamma,trunc,cholesky,coleqnumb,colnfreeparms,colnumb,colsof,det,diag,diag0cnt,el,get,hadamard,I,inv,invsym,issymmetric,J,matmissing,matuniform,mreldif,nullmat,roweqnumb,rownfreeparms,rownumb,rowsof,sweep,trace,vec,vecdiag,autocode,byteorder,c,_caller,chop,abs,clip,cond,e,fileexists,fileread,filereaderror,filewrite,float,fmtwidth,has_eprop,inlist,inrange,irecode,maxbyte,maxdouble,maxfloat,maxint,maxlong,mi,minbyte,mindouble,minfloat,minint,minlong,missing,r,recode,replay,return,s,scalar,smallestdouble,rbeta,rbinomial,rcauchy,rchi2,rexponential,rgamma,rhypergeometric,rigaussian,rlaplace,rlogistic,rnbinomial,rnormal,rpoisson,rt,runiform,runiformint,rweibull,rweibullph,tin,twithin,betaden,binomial,binomialp,binomialtail,binormal,cauchy,cauchyden,cauchytail,chi2,chi2den,chi2tail,dgammapda,dgammapdada,dgammapdadx,dgammapdx,dgammapdxdx,dunnettprob,exponential,exponentialden,exponentialtail,F,Fden,Ftail,gammaden,gammap,gammaptail,hypergeometric,hypergeometricp,ibeta,ibetatail,igaussian,igaussianden,igaussiantail,invbinomial,invbinomialtail,invcauchy,invcauchytail,invchi2,invchi2tail,invdunnettprob,invexponential,invexponentialtail,invF,invFtail,invgammap,invgammaptail,invibeta,invibetatail,invigaussian,invigaussiantail,invlaplace,invlaplacetail,invlogistic,invlogistictail,invnbinomial,invnbinomialtail,invnchi2,invnF,invnFtail,invnibeta,invnormal,invnt,invnttail,invpoisson,invpoissontail,invt,invttail,invtukeyprob,invweibull,invweibullph,invweibullphtail,invweibulltail,laplace,laplaceden,laplacetail,lncauchyden,lnigammaden,lnigaussianden,lniwishartden,lnlaplaceden,lnmvnormalden,lnnormal,lnnormalden,lnwishartden,logistic,logisticden,logistictail,nbetaden,nbinomial,nbinomialp,nbinomialtail,nchi2,nchi2den,nchi2tail,nF,nFden,nFtail,nibeta,normal,normalden,npnchi2,npnF,npnt,nt,ntden,nttail,poisson,poissonp,poissontail,t,tden,ttail,tukeyprob,weibull,weibullden,weibullph,weibullphden,weibullphtail,weibulltail,abbrev,char,collatorlocale,collatorversion,indexnot,plural,plural,real,regexm,regexr,regexs,soundex,soundex_nara,strcat,strdup,string,strofreal,string,strofreal,stritrim,strlen,strlower,strltrim,strmatch,strofreal,strofreal,strpos,strproper,strreverse,strrpos,strrtrim,strtoname,strtrim,strupper,subinstr,subinword,substr,tobytes,uchar,udstrlen,udsubstr,uisdigit,uisletter,ustrcompare,ustrcompareex,ustrfix,ustrfrom,ustrinvalidcnt,ustrleft,ustrlen,ustrlower,ustrltrim,ustrnormalize,ustrpos,ustrregexm,ustrregexra,ustrregexrf,ustrregexs,ustrreverse,ustrright,ustrrpos,ustrrtrim,ustrsortkey,ustrsortkeyex,ustrtitle,ustrto,ustrtohex,ustrtoname,ustrtrim,ustrunescape,ustrupper,ustrword,ustrwordcount,usubinstr,usubstr,word,wordbreaklocale,worcount,acos,acosh,asin,asinh,atan,atanh,cos,cosh,sin,sinh,tan,tanh,},
    % 注释
    morecomment=[l]{//},
    morecomment=[s]{/*}{*/},
    morecomment=[f][\itshape\color{gray}][0]{*}, %当*在首列时,为注释
    morecomment=[f][\itshape\color{gray}][1]{*}, %当*在第二列时,为注释
    morecomment=[f][\itshape\color{gray}][2]{*}, %当*在第三列时,为注释
    % 字符串
    morestring=[s]{`}{'},
    morestring=[s]{"}{"},
    morestring=[s]{"`}{'"},
    morestring=[s]{`"`}{'"'},
} %导入Stata的listings配置文件


%标题页设置
\title{LaTeX:listings 宏包的学习与使用}
\author{王美庭\thanks{王美庭,暨南大学经济与社会研究院,电子信箱:wangmeiting92@gmail.com}} %在独立的提名页中,用致谢命令生成的脚注不附带一条脚注线
\date{\today}

%新定义命令
\renewcommand{\lstlistingname}{源程序}

%LaTeX语言全局设置
\lstset{
	language={[LaTeX]TeX},
	basicstyle={\ttfamily},
	commentstyle={\itshape\color{gray}},
	keywordstyle={\color{blue}},
	morekeywords={heiti,multirow,lstdefinelanguage,lstset,color},
	emph={document,ctexart,tabular,listings,array,morekeywords,morecomment,morestring,language,basicstyle,commentstyle,keywordstyle,stringstyle,emph,breaklines,tabsize,frame,frameround,flexiblecolumns},
	emphstyle={\color{Purple}},
	breaklines=true,
	tabsize=4,
	frame=trBL,
	frameround=fttt,
	flexiblecolumns=true,
}

%超链接全局设置
\hypersetup{
	pdftitle={listings 宏包的学习与使用},
	pdfauthor={王美庭},
	colorlinks=true,
	pdfsubject={这是这个 PDF 的主题,类似于摘要。},
	pdfkeywords={关键词1, 关键词2, ...},
	urlcolor={[RGB]{25, 128, 230}},
	linkcolor={[RGB]{25, 128, 230}}
}


%%----------------------------正文区--------------------------%%
\begin{document}

\maketitle

\tableofcontents

\thispagestyle{empty}

\newpage

\setcounter{page}{1}

\section{引言}
这篇文章会首先介绍 LaTeX 自带的抄录命令和环境,由于自带的抄录命令与环境不能实现代码高亮、行号设置等功能,所以我们接下来会介绍 listings 宏包具有更多特色功能的代码抄录命令和环境。虽然 listings 宏包已经支持了很多程序语言的高亮,但仍然还是有一些程序语言被排除在外,所以本文最后部分给出了如何通过 listings 宏包自定义自己心仪的程序语言的方法\footnote{当本文无法解决你的问题时,我们始终相信该宏包说明文档、搜索引擎以及相关的书籍是你最好的学习资料。}。

\section{LaTeX 自带的代码抄录语句和环境}
单条代码抄录:\verb|\verb<符号><抄录内容><符号>|,即在 \verb|\verb| 之后,起始的符号和末尾的符号一致,两个符号之间的部分将使用打字机字体逐字输出原样输出:
\begin{flushleft}
  \verb"\verb|\textsf{Sans serif font family}|"
  $\rightarrow$
  \verb|\textsf{Sans serif font family}|
\end{flushleft}
宏包 shortvrb 提供了 \verb|\verb| 命令的简写形式,可以使用 \verb|\MakeShortVerb<符号>| 来定义这种简写,以及之后可以用 \verb|\DeleteShortVerb<符号>| 来取消定义,如在导言区定义:
\begin{verbatim}
\usepackage{shortvrb}
\MakeShortVerb|
\end{verbatim}
那么就可以使用竖线 \verb!|! 作为简写方式了:
\MakeShortVerb|
\begin{flushleft}
\verb!verbatim |\LaTeX|!
$\rightarrow$
verbatim |\LaTeX|
\end{flushleft}
\DeleteShortVerb|
使用带星号的命令 \verb|\verb*| 则可以使输出的空格为可见的 \verb*| | :
\begin{flushleft}
  \verb"\verb*|\textsf{Sans serif font family}|"
  $\rightarrow$
  \verb*|\textsf{Sans serif font family}|
\end{flushleft}
大段的抄录则可以使用 \verb|verbatim| 环境:
\begin{flushleft}
\begin{minipage}{0.38\linewidth}
\verb|\begin{verbatim}|\\
\verb|\begin{tabular}{ll}|\\
\verb|  字体族 & 效果\\|\\
\verb|  罗马   & \textrm{rmfamily}\\|\\
\verb|  无衬线 & \textsf{sffamily}\\|\\
\verb|  打字机 & \texttt{ttfamily}\\|\\
\verb|\end{tabular}|\\
\verb|\end{verbatim}|
\end{minipage}
$\rightarrow$
\begin{minipage}{0.45\linewidth}
\begin{verbatim}
\begin{tabular}{ll}
  字体族 & 效果\\
  罗马   & \textrm{rmfamily}\\
  无衬线 & \textsf{sffamily}\\
  打字机 & \texttt{ttfamily}\\
\end{tabular}
\end{verbatim}
\end{minipage}
\end{flushleft}
同样,可以使用带星号的 \verb|verbatim*| 环境输出可见空格:
\begin{flushleft}
\begin{minipage}{0.38\linewidth}
\begin{verbatim}
\begin{verbatim*}
\begin{tabular}{ll}
  字体族 & 效果\\
  罗马   & \textrm{rmfamily}\\
  无衬线 & \textsf{sffamily}\\
  打字机 & \texttt{ttfamily}\\
\end{tabular}
\end{verbatim*}
\end{verbatim}
\end{minipage}
$\rightarrow$
\begin{minipage}{0.45\linewidth}
\begin{verbatim*}
\begin{tabular}{ll}
  字体族 & 效果\\
  罗马   & \textrm{rmfamily}\\
  无衬线 & \textsf{sffamily}\\
  打字机 & \texttt{ttfamily}\\
\end{tabular}
\end{verbatim*}
\end{minipage}
\end{flushleft}


\section{listings 宏包加强版的代码抄录命令与环境}

\subsection{语法与选项}
\label{subsec: syntax and option}

如果我们还要设置代码高亮、行号、背景色等,那 LaTeX 自带的代码抄录语句和环境就不能满足需求了,而 listings 宏包提供的命令和环境可以实现这些需求。其抄录命令的语法如下:
\begin{flushleft}
\verb|\lstinline[参数1=选项, 参数2=选项, ...]<符号><抄录内容><符号>|
\end{flushleft}
类似于前面 \verb!\verb! 可以定义简写形式,该命令也可以定义简写形式,我们可以通过
\begin{flushleft}
\verb!\lstMakeShortInline[参数1=选项, 参数2=选项, ...]<符号>!
\end{flushleft}
定义等价抄录符号,也可以通过 \verb!\lstDeleteShortInline<符号>! 取消前面定义的等价抄录符号。另外其抄录环境的语法如下:
\begin{flushleft}
\begin{verbatim}
\begin{lstlisting}[参数1=选项, 参数2=选项, ...]
    源代码
\end{lstlisting}
\end{verbatim}
\end{flushleft}
其中常用的参数及其选项如下:

\begin{longtable}{@{}p{0.23\linewidth}p{0.73\linewidth}}
	\verb|language| & 设置代码所属语言,语法为 \verb|language=[<dialect>]<language>| ,适用语言可查看其宏包文件 \\
	\verb|basicstyle| & 设置全体的抄录文本字体(可设置字体族、形状、系列、大小、颜色等)。如 \verb|basicstyle={\ttfamily}| \\
	\verb|commentstyle| & 设置注释文本字体,设置方式同 \verb|basicstyle| \\
	\verb|keywordstyle| & 设置关键字文本字体,设置方式同 \verb|basicstyle| \\
	\verb|morekeywords| & 增加系统没有识别的关键词 \\
	\verb|emph| & 增加要被强调的词,如 \verb|emph={document,ctexart}| \\
	\verb|emphstyle| & 设置被强调内容的文字格式,设置方式同 \verb|basicstyle| \\
	\verb|breaklines| & 设置代码是否自动换行,\verb|true| 为真,\verb|false| 为否 \\
	\verb|alsoletter| & 设置不作为单词分隔符的符号,如 \verb|alsoletter={.}| \\
	\verb|stringstyle| & 设置字符串的文字格式,设置方式同 \verb|basicstyle| \\
	\verb|frame| & 边线设置,选项有 \verb|trblTRBL| 的子集、\verb|none|、\verb|leftline|、\verb|topline|、\verb|bottomline|、\verb|lines|、\verb|single|、\verb|shadowbox| \\
	\verb|frameround| & 设置 frame 的四个角分别为尖角还是圆角,如 \verb|frameround={fttt}| ,将设置右上角为尖角,其余角为圆角 \\
	\verb|backgroundcolor| & 设置代码背景色,如 \verb|backgroundcolor={\color{yellow!20}}| \\
	\verb|rulecolor| & 设置 frame 线条颜色,如 \verb|rulecolor={\color{red}}| \\
	\verb|fillcolor| & 设置文字区到边界第一条线的填充色 \\
	\verb|rulesepcolor| & 设置第一条线和第二条线之间的填充色 \\
	\verb|numbers| & 行号位置,选项有 \verb|none|、\verb|left|、\verb|right| \\
	\verb|numberstyle| & 设置行号数字文本字体,设置方式同 \verb|basicstyle| \\
	\verb|tabsize| & 设置 tab 所占空格数 \\
	\verb|title| & 设置标题内容,生成格式为``标题内容'' \\
	\verb|caption| & 设置标题内容,生成格式为``Listing 序号:标题内容'',可重新定义 \verb|\lstlistingname| 命令以替换标题中的``Listing'' \\
	\verb|captionpos| & 设置表格的位置,选项有 \verb|t|、\verb|b| \\
	\verb|framesep| & 控制 listings 和 frame 的距离 \\
	\verb|framexleftmargin| & 控制 listings 和左边 frame 的距离 \\
	\verb|framexrightmargin| & 控制 listings 和右边 frame 的距离 \\
	\verb|framextopmargin| & 控制 listings 和上边 frame 的距离 \\
	\verb|framexbottommargin| & 控制 listings 和下边 frame 的距离 \\
	\verb|flexiblecolumns| & \verb|true| 表示将字符非等宽显示,\verb|false| 表示将字符等宽显示 \\
\end{longtable}

参数除了在使用抄录命令或抄录环境可以被设置之外,还可以在导言区使用 \verb|\lstset| 进行全局设置,其中的可选参数及其选项与前面一致:
\begin{flushleft}
\verb|\lstset{参数1=选项,参数2=选项,...}|
\end{flushleft}
这里值得注意的是,有些参数只在抄录环境中有效,如 \verb|frame|、\verb|title|、\verb|number|、\verb|backgroundcolor| 等。


\subsection{实例 - LaTeX 代码抄录(宏包本身所支持的语言)}
这里以该宏包所支持的 LaTeX 语言为例,实现代码的高亮展示。我们首先在导言区利用 \verb|\lstset| 进行了一些全局设定:
%verbatim环境的段落排版模式类似于flushleft环境
\begin{verbatim}
% LaTeX 语言全局设置
\lstset{
    language={[LaTeX]TeX},
    basicstyle={\ttfamily},
    commentstyle={\itshape\color{gray}},
    keywordstyle={\color{blue}},
    morekeywords={heiti,multirow,lstdefinelanguage,lstset,color},
    emph={document,ctexart,tabular,listings,array},
    emphstyle={\color{Purple}},
    breaklines=true,
    tabsize=4,
    frame=trBL,
    frameround=fttt,
    flexiblecolumns=true,
}
\end{verbatim}
然后在使用抄录命令或抄录环境时我们还可以进行一些额外的设定(如果有需要的话)。首先我们使用抄录命令 \verb|\lstinline| 查看效果:
\begin{flushleft}
\verb!\lstinline|\multicolumn{2}{c}{some texts}|! $\rightarrow$ \lstinline|\multicolumn{2}{c}{some texts}| \\
\verb!\lstinline|\multirow{2}*{some texts}|! $\rightarrow$ \lstinline|\multirow{2}*{some texts}| \\
\verb!\lstinline|\textsf{This is a sentence.}|! $\rightarrow$ \lstinline|\textsf{This is a sentence.}|
\end{flushleft}
然后再使用抄录环境 \verb|lstlisting| 查看效果:
\begin{center}
\begin{verbatim}
\begin{lstlisting}
\begin{tabular}{|l|l|}
    \hline
    \heiti 字体族 & \heiti 效果                     \\\hline
    罗马          & \textrm{Roman font family}      \\\hline
    无衬线        & \textsf{Sans serif font family} \\\hline
    打字机        & \texttt{Typewriter font family} \\\hline
\end{tabular}
\end{lstlisting}
\end{verbatim}
$\downarrow$ \\[0.25em]
\begin{lstlisting}
\begin{tabular}{|l|l|}
	\hline
	\heiti 字体族 & \heiti 效果                     \\\hline
	罗马          & \textrm{Roman font family}      \\\hline
	无衬线        & \textsf{Sans serif font family} \\\hline
	打字机        & \texttt{Typewriter font family} \\\hline
\end{tabular}
\end{lstlisting}
\end{center}
最后,我们来抄录一份比较完整的 LaTeX 代码:
\begin{verbatim}
\begin{lstlisting}
%The introduction area(导言区)
\documentclass[UTF8]{ctexart}
\usepackage{array}
\usepackage{listings}

%The body area(正文区)
\begin{document}
    \section{英文字体}
    \begin{tabular}{ll}
        \heiti 字体族 & \heiti 效果                     \\
        罗马          & \textrm{Roman font family}      \\
        无衬线        & \textsf{Sans serif font family} \\
        打字机        & \texttt{Typewriter font family}
    \end{tabular}

    This is a sentence about English.

    这是一句中文(在 ttfamily 中,中文字体变成了仿宋)。
\end{document}
\end{lstlisting}
\end{verbatim}
\begin{center} $\downarrow$ \end{center}
\begin{lstlisting}
% The introduction area(导言区)
\documentclass[UTF8]{ctexart}
\usepackage{array}
\usepackage{listings}

% The body area(正文区)
\begin{document}
\section{英文字体}
\begin{tabular}{ll}
	\heiti 字体族 & \heiti 效果                     \\
	罗马          & \textrm{Roman font family}      \\
	无衬线        & \textsf{Sans serif font family} \\
	打字机        & \texttt{Typewriter font family}
\end{tabular}

This is a sentence about English.

这是一句中文(在 ttfamily 中,中文字体变成了仿宋)。
\end{document}
\end{lstlisting}

\subsection{实例 - Stata 代码抄录(宏包本身不支持的语言)}
肯定有读者发现了,这么好用的宏包竟然不支持 Stata 语言的代码,那就自己定义一个吧。过程还是很简单的,如下面所示\footnote{不像前面,从这里开始,所有的 LaTeX 代码都将进入高亮模式。前面之所以黑白模式和高亮模式交替进行,是为了让读者能更好的理解 listings 宏包的运作方式。}:
\begin{lstlisting}
\lstdefinelanguage{Stata}{
	% 首要关键字
	morekeywords=[1]{regress, reg, summarize, sum, ...}, %定义一级关键字
	% 函数关键字
	morekeywords=[2]{ustrregexm, ustrregexra, ...}, %定义二级关键字
	% 注释
	morecomment=[l]{//}, %定义\\之后为注释语句
	morecomment=[s]{/*}{*/}, %定义/*和*/之间为注释语句
	morecomment=[f][\itshape\color{gray}][0]{*}, %当*在首列时,为注释
	morecomment=[f][\itshape\color{gray}][1]{*}, %当*在第二列时,为注释
	morecomment=[f][\itshape\color{gray}][2]{*}, %当*在第三列时,为注释
	% 字符串
	morestring=[s]{`}{'}, %定义`和'之间为字符串,下面类似
	morestring=[s]{"}{"},
	morestring=[s]{"`}{'"},
	morestring=[s]{`"`}{'"'},
}
\end{lstlisting}

在这里的 Stata 语言定义中,我们将最经常用的 \verb|regress| 、\verb|summarize| 等定义为一级关键字,将常用的函数(含矩阵函数、数学函数、字符串函数等)定义为二级关键字。另外,这里之所以将宏(也称展元)定义为字符串,是因为该宏包中没有额外的办法来专门设定宏了\footnote{如果读者不喜欢这条设定,则删除对应条目即可。个人之所以这么设定,是因为这样做还是可以让宏代码起到一定的突出效果。}。

Stata 语言的完全体定义见个人提供的 listings-stata.tex 文件\footnote{里面肯定还有很多关键字和函数是没有的,有兴趣的读者可以自行添加,打造属于自己的 listings-stata.tex 文件。},读者只需将其放在当前目录下,并在主文件导言区使用 \lstinline[alsoletter={-}]|% Description: Stata language definition for listings
% Author: WANG Meiting, PhD, Institute for Economic and Social Research, Jinan University
% Email: wangmeiting92@gmail.com
% Created on Sep 15, 2020
%
% Inspired by "https://git.io/JU8tn" and "https://git.io/JU8tG"

\lstdefinelanguage{Stata}{
    % 首要关键字
    morekeywords=[1]{regress,reg,summarize,sum,display,di,generate,gen,bysort,use,import,delimited,predict,quietly,probit,margins,test,forvalues,foreach,set,sysuse,graph,local,global,export,twoway,replace,append,clear,if,else,dis,order,mat,matrix,mata,excel,cd,logit,aggregate,array,boolean,break,byte,case,catch,class,colvector,complex,const,continue,default,delegate,delete,do,double,eltypedef,end,enum,explicit,external,float,for,friend,function,goto,inline,int,long,namespace,new,numeric,NULL,operator,orgtypedef,pointer,polymorphic,pragma,private,protected,public,quad,real,return,rowvector,scalar,short,signed,static,strL,string,struct,super,switch,template,this,throw,transmorphic,try,typedef,typename,union,unsigned,using,vector,version,virtual,void,volatile,while,in,define,rown,rownames,coln,colnames,list,corr,ttest,tokenize},
    % 函数关键字
    morekeywords=[2]{bofd,Cdhms,Chms,Clock,clock,Cmdyhms,Cofc,cofC,Cofd,cofd,daily,date,day,dhms,dofb,dofC,dofc,dofh,dofm,dofq,dofw,dofy,dow,doy,halfyear,halfyearly,hh,hhC,hms,hofd,hours,mdy,mdyhms,minutes,mm,mmC,mofd,month,monthly,msofhours,msofminutes,msofseconds,qofd,quarter,quarterly,seconds,ss,ssC,tC,tc,td,th,tm,tq,tw,week,weekly,wofd,year,yearly,yh,ym,yofd,yq,yw,abs,ceil,cloglog,comb,digamma,exp,expm1,floor,int,invcloglog,invlogit,ln,ln1m,ln,ln1p,ln,lnfactorial,lngamma,log,log10,log1m,log1p,max,min,mod,reldif,round,sign,sqrt,trigamma,trunc,cholesky,coleqnumb,colnfreeparms,colnumb,colsof,det,diag,diag0cnt,el,get,hadamard,I,inv,invsym,issymmetric,J,matmissing,matuniform,mreldif,nullmat,roweqnumb,rownfreeparms,rownumb,rowsof,sweep,trace,vec,vecdiag,autocode,byteorder,c,_caller,chop,abs,clip,cond,e,fileexists,fileread,filereaderror,filewrite,float,fmtwidth,has_eprop,inlist,inrange,irecode,maxbyte,maxdouble,maxfloat,maxint,maxlong,mi,minbyte,mindouble,minfloat,minint,minlong,missing,r,recode,replay,return,s,scalar,smallestdouble,rbeta,rbinomial,rcauchy,rchi2,rexponential,rgamma,rhypergeometric,rigaussian,rlaplace,rlogistic,rnbinomial,rnormal,rpoisson,rt,runiform,runiformint,rweibull,rweibullph,tin,twithin,betaden,binomial,binomialp,binomialtail,binormal,cauchy,cauchyden,cauchytail,chi2,chi2den,chi2tail,dgammapda,dgammapdada,dgammapdadx,dgammapdx,dgammapdxdx,dunnettprob,exponential,exponentialden,exponentialtail,F,Fden,Ftail,gammaden,gammap,gammaptail,hypergeometric,hypergeometricp,ibeta,ibetatail,igaussian,igaussianden,igaussiantail,invbinomial,invbinomialtail,invcauchy,invcauchytail,invchi2,invchi2tail,invdunnettprob,invexponential,invexponentialtail,invF,invFtail,invgammap,invgammaptail,invibeta,invibetatail,invigaussian,invigaussiantail,invlaplace,invlaplacetail,invlogistic,invlogistictail,invnbinomial,invnbinomialtail,invnchi2,invnF,invnFtail,invnibeta,invnormal,invnt,invnttail,invpoisson,invpoissontail,invt,invttail,invtukeyprob,invweibull,invweibullph,invweibullphtail,invweibulltail,laplace,laplaceden,laplacetail,lncauchyden,lnigammaden,lnigaussianden,lniwishartden,lnlaplaceden,lnmvnormalden,lnnormal,lnnormalden,lnwishartden,logistic,logisticden,logistictail,nbetaden,nbinomial,nbinomialp,nbinomialtail,nchi2,nchi2den,nchi2tail,nF,nFden,nFtail,nibeta,normal,normalden,npnchi2,npnF,npnt,nt,ntden,nttail,poisson,poissonp,poissontail,t,tden,ttail,tukeyprob,weibull,weibullden,weibullph,weibullphden,weibullphtail,weibulltail,abbrev,char,collatorlocale,collatorversion,indexnot,plural,plural,real,regexm,regexr,regexs,soundex,soundex_nara,strcat,strdup,string,strofreal,string,strofreal,stritrim,strlen,strlower,strltrim,strmatch,strofreal,strofreal,strpos,strproper,strreverse,strrpos,strrtrim,strtoname,strtrim,strupper,subinstr,subinword,substr,tobytes,uchar,udstrlen,udsubstr,uisdigit,uisletter,ustrcompare,ustrcompareex,ustrfix,ustrfrom,ustrinvalidcnt,ustrleft,ustrlen,ustrlower,ustrltrim,ustrnormalize,ustrpos,ustrregexm,ustrregexra,ustrregexrf,ustrregexs,ustrreverse,ustrright,ustrrpos,ustrrtrim,ustrsortkey,ustrsortkeyex,ustrtitle,ustrto,ustrtohex,ustrtoname,ustrtrim,ustrunescape,ustrupper,ustrword,ustrwordcount,usubinstr,usubstr,word,wordbreaklocale,worcount,acos,acosh,asin,asinh,atan,atanh,cos,cosh,sin,sinh,tan,tanh,},
    % 注释
    morecomment=[l]{//},
    morecomment=[s]{/*}{*/},
    morecomment=[f][\itshape\color{gray}][0]{*}, %当*在首列时,为注释
    morecomment=[f][\itshape\color{gray}][1]{*}, %当*在第二列时,为注释
    morecomment=[f][\itshape\color{gray}][2]{*}, %当*在第三列时,为注释
    % 字符串
    morestring=[s]{`}{'},
    morestring=[s]{"}{"},
    morestring=[s]{"`}{'"},
    morestring=[s]{`"`}{'"'},
}| 将其导入即可。之后的操作就和使用 listings 内置的程序语言一样了。

首先在导言区利用 \lstinline|\lstset| 进行全局设置:
\begin{lstlisting}
% Stata 语言全局设置
\lstset{
	language=Stata,
	basicstyle={\ttfamily},
	commentstyle={\itshape\color{gray}}, %设置注释的文字格式
	keywordstyle={[1]\color{Magenta}}, %设置首要关键字的文字格式
	keywordstyle={[2]\color{blue}}, %设置函数关键字的文字格式
	stringstyle={\color{Purple}}, %设置字符串的文字格式
	emph={}, %使前面 LaTeX 语言所设定的 emph 失效
	breaklines=true,
	tabsize=4,
	frame=trBL,
	frameround=fttt,
	flexiblecolumns=true,
}
\end{lstlisting}

%Stata语言全局设置
\lstset{
	language=Stata,
	basicstyle={\ttfamily},
	commentstyle={\itshape\color{gray}}, %设置注释的文字格式
	keywordstyle={[1]\color{Magenta}}, %设置首要关键字的文字格式
	keywordstyle={[2]\color{blue}}, %设置函数关键字的文字格式
	stringstyle={\color{Purple}}, %设置字符串的文字格式
	emph={}, %使前面LaTeX语言所设定的emph失效
	breaklines=true,
	tabsize=4,
	frame=trBL,
	frameround=fttt,
	flexiblecolumns=true,
}

然后就可以让 Stata 的代码在 LaTeX 排版中高亮起来了,比如命令行高亮\footnote{这里没有直接给出抄录命令或抄录环境源码,其语法与选项参见 \ref{subsec: syntax and option} 节,下同。}:
\begin{flushleft}
\lstinline|matrix define A = matuniform(3,3)|
\end{flushleft}
命令块高亮:
\begin{lstlisting}
* This is the comment(这是注释)
 * This is the comment(这是注释)
  * This is the comment(这是注释)
sysuse auto.dta, clear
sum price mpg weight trunk //This is the comment at the end of the sentence
ttest price, by(foreign)
corr price mpg weight trunk //求相关系数
reg price mpg trunk weight
logit foreign price mpg displacement

* 矩阵运算
mat define A = (1,2,3\4,5,6\7,8,9)
mat rownames A = row1 row2 row3
mat B = J(3,3,1)
mat C = A * B
/*
这是多行注释。
There are multiple lines of comments here.
*/

* 字符串处理
local aa "This is a string"
local bb = ustrregexra("`xx'","\w{4}","")
local cc = ustrregexra(`"`xx'"',"\w{2}","")

* 循环程序显示拆分的字符串
local xx "This is a string"
tokenize `xx'
local i = 1
while "``i''" != "" {
	dis in y "``i++''"
}

* 循环程序求和
local sum_i = 0
forvalues i = 1/100 {
	local sum_i = `sum_i' + `i'
}
dis in y `sum_i'
\end{lstlisting}

这样设置了一下之后,感觉在 LaTeX 中排版的代码看起来顺眼多了。

\subsection{其他注意事项}

%LaTeX语言全局设置
\lstset{
	language={[LaTeX]TeX},
	basicstyle={\ttfamily},
	commentstyle={\itshape\color{gray}},
	keywordstyle={\color{blue}},
	morekeywords={heiti,multirow,lstdefinelanguage,lstset,color},
	emph={document,ctexart,tabular,listings,array,morekeywords,morecomment,morestring,language,basicstyle,commentstyle,keywordstyle,stringstyle,emph,breaklines,tabsize,frame,frameround,flexiblecolumns},
	emphstyle={\color{Purple}},
	breaklines=true,
	tabsize=4,
	frame=trBL,
	frameround=fttt,
	flexiblecolumns=true,
}

在对 \lstinline|\lstset| 进行设置时,还有一些其他要注意的细节:
\begin{itemize}
	\item 在同一篇文章中,一种语言定义的 \verb|morekeywords| 不会影响对另外一种语言的 \verb|keywords| 产生影响。如在同一篇文章中,LaTeX 语言额外定义了 \verb|multirow| 为关键字,则这个关键字只会在 LaTeX 语言的抄录中高亮,而不会在 Matlab 语言的抄录中高亮。
	\item 在同一篇文章中,一种语言定义的 \verb|emph| 会影响到另外一种语言。如在同一篇文章中,LaTeX 语言定义了 \verb|tabular| 为强调词语,则这个强调词语会过渡到下一个语言。当然,我们一般不愿意看到这样的结果,此时我们只需对 \verb|emph| 进行重新设定即可。之所以可以这么做,是因为对于那些可以传递的参数,后面\lstset{}的设置,会替换前面\lstset{}对于相同参数的设置。
	\item \verb|morekeywords| 需要和 \verb|language| 放在一起,否则会失效(无论是针对自定义的还是现有的语言)。当然,在抄录命令和抄录环境中的局部 \verb|morekeywords| 设定不受此影响。
\end{itemize}




\section*{参考资料}
\addcontentsline{toc}{section}{参考资料} %将其添加进目录
\markright{参考资料} %将其添加进页眉
\noindent
\href{https://www.ctan.org/pkg/listings}{CTAN: listings – Typeset source code listings using \LaTeX{}} \\
\href{https://github.com/satejsoman/stata-lstlisting}{GitHub: satejsoman/stata-lstlisting} \\
\href{https://gist.github.com/mcaceresb/b40d6059cf66cc73423f4ddf3f72acda}{GIST-GitHub: mcaceresb/listings-stata.tex} \\
刘海洋:《\LaTeX{} 入门》,第 2.2.5 节,抄录与代码环境 \\
胡伟:《\LaTeXe{} 完全学习手册(第二版)》,第 2.11.9 节,抄录环境和命令

\end{document}
